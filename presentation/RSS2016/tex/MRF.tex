%%%%%%%%%%%%%%%%%%%%%%%%%%%%%%%%%%%%%%%%%%%%%%%%%%%%%%%%%%%%%%%
%
% Welcome to Overleaf --- just edit your LaTeX on the left,
% and we'll compile it for you on the right. If you give
% someone the link to this page, they can edit at the same
% time. See the help menu above for more info. Enjoy!
%
% Note: you can export the pdf to see the result at full
% resolution.
%
%%%%%%%%%%%%%%%%%%%%%%%%%%%%%%%%%%%%%%%%%%%%%%%%%%%%%%%%%%%%%%%
\documentclass{beamer}

\usepackage{tikz}

\usepackage{verbatim}
\usetikzlibrary{arrows,shapes}


\begin{document}
\begin{comment}
:Title: Prim's algorithm
:Tags: Beamer, Layers, Foreach, Graphs
:Use page: 6


A step by step example of the `Prim's algorithm`_ for finding the `minimum
spanning tree`_. Animated using Beamer 
overlays.

.. _Prim's algorithm: http://en.wikipedia.org/wiki/Prim%27s_algorithm
.. _Minimum spanning tree: http://en.wikipedia.org/wiki/Minimum_spanning_tree

| Source: Adapted from an example on Wikipedia_

.. _Wikipedia: http://en.wikipedia.org/wiki/Prim%27s_algorithm
\end{comment}

% Declare layers
\pgfdeclarelayer{background}
\pgfsetlayers{background,main}

\begin{frame}
%% Adjacency matrix of graph
%% \  a  b  c  d  e  f  g
%% a  x  7     5
%% b  7  x  8  9  7
%% c     8  x     5
%% d  5  9     x 15  6
%% e     7  5 15  x  8  9
%% f           6  8  x 11
%% g              9  11 x

\tikzstyle{vertex}=[circle,fill=white,minimum size=20pt,inner sep=0pt]
\tikzstyle{selected vertex} = [vertex, fill=red!24]
\tikzstyle{edge} = [draw,thick,-]
\tikzstyle{weight} = [font=\small]
\tikzstyle{selected edge} = [draw,line width=5pt,-,red!50]
\tikzstyle{ignored edge} = [draw,line width=5pt,-,black!20]

\begin{figure}
\begin{tikzpicture}[scale=1.8, auto,swap]
    % Draw a 7,11 network
    % First we draw the vertices
    \foreach \pos/\name in {{(0,3)/X_{0,0}}, {(0,2)/X_{1,0}}, {(0,1)/X_{2,0}}, {(0,0)/X_{3,0}},
    						{(1,3)/X_{0,1}}, {(1,2)/X_{1,1}}, {(1,1)/X_{2,1}}, {(1,0)/X_{3,1}},
    						{(2,3)/X_{0,2}}, {(2,2)/X_{1,2}}, {(2,1)/X_{2,2}}, {(2,0)/X_{3,2}},
                            {(3,3)/X_{0,3}}, {(3,2)/X_{1,3}}, {(3,1)/X_{2,3}}, {(3,0)/X_{3,3}}}
        \node[vertex] (\name) at \pos {$\name$};
    % Connect vertices with edges and draw weights
    \foreach \source/ \dest /\weight in {X_{0,0}/X_{1,0}/, X_{0,0}/X_{0,1}/,
    									 X_{1,0}/X_{2,0}/, X_{1,0}/X_{1,1}/,
                                         X_{2,0}/X_{3,0}/, X_{2,0}/X_{2,1}/,
                                         X_{3,0}/X_{3,1}/,
                                         X_{0,1}/X_{0,2}/, X_{0,1}/X_{1,1}/,
                                         X_{1,1}/X_{2,1}/, X_{1,1}/X_{1,2}/,
                                         X_{2,1}/X_{3,1}/, X_{2,1}/X_{2,2}/,
                                         X_{3,1}/X_{3,2}/,
                                         X_{0,2}/X_{0,3}/, X_{0,2}/X_{1,2}/,
                                         X_{1,2}/X_{2,2}/, X_{1,2}/X_{1,3}/,
                                         X_{2,2}/X_{3,2}/, X_{2,2}/X_{2,3}/,
                                         X_{3,2}/X_{3,3}/,
                                         X_{0,3}/X_{1,3}/,
                                         X_{1,3}/X_{2,3}/,
                                         X_{2,3}/X_{3,3}/}
        \path[edge] (\source) -- node[weight] {$\weight$} (\dest);
\end{tikzpicture}
\end{figure}

\end{frame}

\end{document}